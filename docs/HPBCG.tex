\documentclass{article}
\usepackage{eurosym} % \euro{} \geneuro{} \geneuronarrow{} \geneurowide
\usepackage[french]{babel}
\usepackage[T1]{fontenc}
\usepackage[latin1]{inputenc}
\usepackage{url}
%\usepackage[utf8]{inputenc}
\usepackage{graphicx}
\title{HPBCG Documentation\\
High Performance Binary Code Generation}
\author{Henri-Pierre Charles}
\newcommand{\hpbcg}{\textbf{HPBCG}\ }
\begin{document}
\maketitle

\section{Introduction}

\hpbcg is a tool which help to build binary code generator.

Actually (january 2009) computer architecture reach a complexity
point which leed to
\begin{itemize}
\item compiler which are unable to vectorize or use multimedia
  instructions easily
\item a bad use of huge register set. Compiler are still using
  algorithm allocator which came from ages where register are rare.
\item data are the main important parameter that actual compiler
  cannot take into account because code generation is done at static
  compile time.
\end{itemize}

\section{Actual status}

\begin{description}
\item[Cell spu] preliminar
\end{description}

\section{Using \hpbcg}

\subsection{Installing \hpbcg}

\hpbcg sould work on any reasonnable unix like target. The requirements
are :
\begin{description}
\item[antlr] The parser is implemented with \url{antlr.org}
\item[java] As antlr is writed in java. But java is not needed at
  run-time, only at static compile time.
\end{description}


\subsection{Using hpbcg}

\section{Compilettes examples}

\subsection{Simple compilette}


\section{Porting \hpbcg}

Porting \hpbcg to a new architecture should be as simple than the
architectural model you plan to target.

\subsection{Architecture description}

\subsection{HPBCG Parser}

\section{Assembly languages}

This part is devoted to different assembly langaguages that \hpbcg

\subsection{cell-spu}

\begin{description}
\item[Integer register names] one of
  \begin{itemize}
  \item \verb|$lr|, \verb|$sp|
  \item \verb|$0| .. \verb|$32|
  \end{itemize}
\end{description}

\subsection{cell-ppu}

\begin{description}
\item[Integer register name]  one of
  \begin{itemize}
  \item \verb|0| .. \verb|32|
  \end{itemize}
\end{description}

\subsection{power}
\begin{description}
\item[Integer register name]  one of
  \begin{itemize}
  \item \verb|r0| .. \verb|r32|
  \end{itemize}
\end{description}


\end{document}

% Local Variables: ***
% compile-command:"MonMake HPBCG.pdf" ***
% End: ***

\documentclass{article}
\usepackage{url}
%\usepackage[utf8]{inputenc}
\usepackage{graphicx}
\title{HPBCG Documentation\\
High Performance Binary Code Generation}
\author{Henri-Pierre Charles}
\newcommand{\hpbcg}{\textbf{HPBCG}\ }
\begin{document}
\maketitle
\tableofcontents{}

\section{Introduction}

\hpbcg is a tool which help to build binary code generator.

Actually (january 2009) computer architecture reach a complexity
point which leed to
\begin{itemize}
\item compiler which are unable to vectorize or use multimedia
  instructions easily
\item a bad use of huge register set. Compiler are still using
  algorithm allocator which came from ages where register are rare.
\item data are the main important parameter that actual compiler
  cannot take into account because code generation is done at static
  compile time.
\end{itemize}

\section{Actual status}

\begin{description}
\item[Cell spu] Working draft 
  \begin{itemize}
  \item simple-multiply fonctionnal
  \end{itemize}
\item[Itanium]  Preliminar
\end{description}

\section{Using \hpbcg}

\subsection{Installing \hpbcg}

\hpbcg sould work on any reasonnable unix like target. The
requirements are :
\begin{description}
\item[antlr] The parser is implemented with \url{antlr.org}
\item[java] As antlr is writed in java. But java is not needed at
  run-time, only at static compile time.
\end{description}

\hpbcg contain two parts :
\begin{description}
\item[Architecture description] contains the architecture description
  and the parser used to generate the macro instructions. This part is
  in the \texttt{src/isatobcg} directory
\item[Parser] contains the parser in charge of the translation from
  the \texttt{.hg} file to the \texttt{.c} file. This par is in the
  \texttt{src/parser} directory.
\end{description}

\subsection{Using hpbcg}

\section{Compilettes examples}

\subsection{simple-multiply}

This example is very simple, it's just a proof of concept.

The obtained result is
\begin{verbatim}
turner:simple-multiply/>./simple-multiply-cell 42
Code generation for multiply value 42
Code generated
  1   2   3   4   5   6   7   8   9  10 
 42  84 126 168 210 252 294 336 378 420 
\end{verbatim}

The \texttt{simple-multiply} program should generate a specialized
version of a very simple program. The non specialized version is :
\begin{verbatim}
int multiply (int a, int b)
{
  return a * b;
}
\end{verbatim}

This compilette will specialize this code with an ``optimized''
version at run-time. For example the previous code will be specialized
as
\begin{verbatim}
int multiply (int a)
{
  return a * 42;
}
\end{verbatim}

\subsubsection{Cell version}

Use the command \texttt{make cell} to build the program.

The \texttt{cell} version contain 2 files :
\begin{description}
\item[simple-worker-cell.c] contain the initial SPU code. It will
  \begin{enumerate}
  \item download the binary code in a buffer
  \item use this buffer as a function
  \item call this function for all incoming parameter
  \end{enumerate}
\item[simple-multiply-cell.hg] is the code for the PPU. It will
  \begin{enumerate}
  \item generate a specialized code depending on the data given by the user.
  \item sent it to the worker
  \item use the worker 10 times for printing a array of multiplied values
  \end{enumerate}
\end{description}

\subsubsection{Itanium version}

Use the command \texttt{make ia64} to build the program.

\section{Porting \hpbcg}

Porting \hpbcg to a new architecture should be as simple than the
architectural model you plan to target.

\subsection{Architecture description}

The actual version contain processor description for 
\begin{description}
\item[cell.isa] This file contain all SPU instruction description
\item[ia64.isa] This file contain all instruction set description
\end{description}

A processor description file should contain
\begin{description}
\item[Comments] A comments line is a line starting with \#
\item[Arch length] A line containing the architecture name and the bit
  lenght of one instruction
\item[Instruction description] This part
\end{description}

\subsection{HPBCG Parser}

\section{Assembly languages}

This part is devoted to different assembly langaguages that \hpbcg

\subsection{cell-spu}

\begin{description}
\item[Integer register names] one of
  \begin{itemize}
  \item \verb|$lr|, \verb|$sp|
  \item \verb|$0| .. \verb|$32|
  \end{itemize}
\item[Calling convention] 
\end{description}

\subsection{cell-ppu}

\begin{description}
\item[Integer register name]  one of
  \begin{itemize}
  \item \verb|0| .. \verb|32|
  \end{itemize}
\item[Calling convention] 
\end{description}

\subsection{power}
\begin{description}
\item[Integer register name]  one of
  \begin{itemize}
  \item \verb|r0| .. \verb|r32|
  \end{itemize}
\item[Calling convention] 
\end{description}

\section{Reporting bug}

Please mail your comments to \url{mailto:hpc@prism.uvsq.fr}

\end{document}

% Local Variables: ***
% compile-command:"MonMake HPBCG.pdf" ***
% End: ***
